\documentclass{article}
\usepackage{graphicx}
\usepackage{latexsym}
\usepackage[french]{babel}
\usepackage[utf8]{inputenc}
\usepackage[T1]{fontenc}
\usepackage{listings}

\title{Rapport du projet C++ : Lancer de rayons}
\author{Mathieu \textsc{Mari} \and Xavier \textsc{Montillet}}

\begin{document}

\title
\tableofcontents
	

\section{Introduction}
Dans ce nouveau projet, nous nous sommes intéressé à la technique du lancer de rayons et nous l'avons implémenter en C++.
Cette technique permet de générer des images 3D à partir de la déscription des objets présent dans la scène ainsi que de la position de la caméra. Elle a son utilité dans les logiciels d'images de synthèse, la crétion de jeux vidéos, la simulation numérique\dots  	


	
\end{document}
